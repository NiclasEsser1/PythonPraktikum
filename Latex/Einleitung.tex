\chapter{Einleitung}
Die Programmiersprache Python wurde von Guido van Rossmann mit dem Ziel entwickelt, um einen einfachen Einstieg in die Welt des Programmierens zu ermöglichen. Tatsächlich wurde die Python zu Schulungs- und Lernzwecke entworfen, denn sie zeichnet sich besonders aus lesbaren und übersichtlichen Code zu schreiben. Der Programmierer ist durch die vorgebene Syntax - auf welcher wir später noch zu sprechen kommen - gezwungen, einen sauberen Stil einzuhalten. Mittlerweile hat Python den vierten Platz der weitverbreitesten Programmiersprachen besetzt, unter den \textit{Skriptsprachen} ist sie sogar die beliebteste. Einer von vielen Gründen der Python so mächtig macht, denn die große Community entwickelt bereits seit fast drei Jahrzenten verschiedenste Bibliotheken mit unterschiedlisten Anwendungsbereichen, es sind (fast) keine Grenzen gesetzt. Beispielweise können Webanwendungen, Datenbank-Verwaltungssysteme, Netzwerkverbindungen, Sensor/Aktorsysteme und mittlerweile sogar Microcontroller progreammiert werden. 

\section{Funktionweise}
Python ist wie erwähnt eine Skriptsprache, die sich dadurch auszeichnen, dass der Programmcode nicht kompiliert werden muss, sondern zur Laufzeit \textit{interpretiert} wird. Deshalb ist auch häufig die Rede von Interpretersprachen. Diese unterscheiden sich deutlich von komilierbaren Programmiersprachen wie z.B. Java oder C/C++, bei welchen der Programmcode in ausführbare Dateien (z.B .exe) übersetzt/kompiliert wird. Kompilierte Dateien funktionieren zwar vollständig autonom, aber sind deutlich unflexibler im Vergleich zu Skript- bzw. Interpretersprachen, wie sich später noch herausstellen wird. 

Der Python-Interpreter kann auf einfache Weise um neue Funktionen und Datentypen erweitert werden, die in C oder C++ (oder andere Sprachen, die sich von C ausführen lassen) implementiert sind. Auch als Erweiterungssprache für anpassbare Applikationen ist Python hervorragend geeignet.

\subsection{Datentypen}
Wie jede andere Programmiersprache besitzt auch Python Datentypen, die vom Programmierer allerdings \underline{nicht} deklariert werden müssen. Wir führen einen kleinen Vergleich an dieser Stelle zwischen der Deklaration eines Strings\footnote{Ein String ist eine Zeichenkette die aus Buchstaben, Symbolen, Nummern etc. bestehen kann, jedes einzelne Zeichen ist ein sogenannter \textit{Character} (char).} in Java un der eines in Python.


\begin{lstlisting}
//Java
String str = "1 String in Java!";
char c = "C";
int zahl = 10;
\end{lstlisting}
\begin{lstlisting}
//Python
str = "1 String in Python"
c = "C "
zahl = 10
\end{lstlisting}